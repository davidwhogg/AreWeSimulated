% Copyright 2021 the authors. All rights reserved.

% To-do items
% -----------
% - make sure I make the point that the sim hypothesis DOES make predictions if there is any finite probability on the simulation being imperfect
% - make sure I make the point that this, in turn, makes the existence of God potentially observable
% - make sure I clearly make the distinction between a perfect simulation and an approximate simulation
% - make sure I clearly make the distinction between an intentionally constructed simulation and a naturally occurring simulation, if that distinction can exist?
% - connection to the anthropic principle: The Universe has a purpose. So maybe the anthropic principle is different in a simulated universe?
% - need a literature review of relevant physics papers.
% - acknowledgements: Kleban, Ness, many students through the years.
% - search for all ... or HOGG or DATE or CITE.

\documentclass[12pt,letterpaper]{article}
\usepackage{url}
\usepackage{xcolor}

% adjust typesetting
\makeatletter
\renewcommand\section{\@startsection {section}{1}{\z@}%
  {-3.25ex \@plus -1ex \@minus -.2ex}%
  {1.5ex \@plus .2ex}%
  {\raggedright\normalfont\large\bfseries}}
\makeatother
\addtolength{\topmargin}{-0.80in}
\addtolength{\textheight}{1.90in}
\linespread{1.08333} % 10/13 spacing
\raggedbottom\sloppy\sloppypar\frenchspacing
\pagestyle{myheadings}
\markright{\textsf{\upshape\color{gray}%
Hogg \& Seraphim --- Is the simulation hypothesis a physics question?}\hfill}
\thispagestyle{empty}

% text macros
\newcommand{\documentname}{\textsl{Note}}
\newcommand{\sectionname}{Section}
\newcommand{\secref}[1]{\sectionname~\ref{#1}}
\newcommand{\foreign}[1]{\textsl{#1}}
\newcommand{\eg}{\foreign{e.~g.}}

\begin{document}

\section*{Does the simulation hypothesis lead to physics questions?}

\medskip\noindent
\textbf{David W. Hogg}
{\footnotesize~\\\url{https://orcid.org/0000-0003-2866-9403}\\Center for Cosmology and Particle Physics, Department of Physics, New York University}

\medskip\noindent
\textbf{Paula Seraphim}
{\footnotesize~\\College of Arts and Sciences, New York University}

\paragraph{Abstract:}
Do we live inside a computer simulation?
This question is of interest to many students of physics, but is it really a physics question?
One possible answer is that if the simulation hypothesis makes predictions---even vague or conditional predictions---for the outcomes of experimental physics programs, then yes, it is a physics question.
There are indeed a number of such predictions, depending on whether the simulation is classical or quantum, how the simulators or simulating universe intervenes in the simulation, what kinds of approximations are in play, and other engineering considerations (such as bugs).
For example, if the simulation is, or involves, exact quantum simulation of quantum states in this world, and it is being observed by (or interacting with) any external system, then there might be observable violations of unitarity or quantum relationships.
For another, if the simulation is approximate or spatially limited, the limits of the approximations might lead to observable effects.
No particular observable consequences of the simulation are predicted at high confidence because they are all conditional on details of the simulation technology and objectives, but there are nonetheless generic signatures that could be used to guide experimental searches; in this way the search for evidence for a simulation is somewhat akin to searches for the particle nature of the dark matter.
One epistemological point is that if we do indeed live in a computer simulation of the kind envisioned by movies and philosophers, then quite literally our maker (or something like god) is potentially subject to discovery by experimentation.
This \documentname\ violates Hinchliffe's Law.

\section{Introduction}

When one spends some time at the beach, watching the waves slowly break and crash onto the sand in foamy washes, with sandpipers running to grab tiny insects displaced by the water in the sand, one must to conclude that the external simulators---if we do indeed live in a computer simulation---are spending one ridiculous amount of computing on the shoreline.
Or maybe the externals are only spending their computation on the part of the shoreline that one is currently looking at, and only for that time, but still?
Or maybe it is a full quantum simulation, in which case the externals are spending way more computation on \emph{each individual grain of sand} than whatever is required to make those waves crash and sandpipers run.
But of course we don't know the intentions and capabilities of the externals.
Maybe computation is really cheap in the external world, and maybe the questions being answered by this simulation are of enormous importance there.
So the question arises: Is it conceivable that we live in a computer simulation?
And what do we even mean by that idea?

The point of this \documentname{} is \emph{not} to answer the question ``Do we live in a simulation?''.
It is to answer the question ``Is this a physics question, worthy of consideration by physicists?''.
Oddly, we think the answer to this latter question is ``yes'':
The simulation hypothesis is important enough to justify consideration, and it can and does lead to some natural kinds of (perhaps vague) predictions for physical experiments in physics laboratories and observatories.
That is, the simulation hypothesis is an idea that can inspire searches for new physics or anomalies, in much the same way that other high-level ideas in physics, such as inflation, string theory, or unification have and do inspire experimental searches.

Here we discuss some of the relevant questions and definitions, with a goal of starting this conversation.
Because we encounter these questions so much in the undergraduate-education context, we have tried to set the level for an advanced undergraduate studying physics, philosophy, neuroscience, or a related field.
We are aware that many physicists consider the simulation hypothesis ridiculous or fantastical.
But our view is that things that are legitimately interesting to students of physics are themselves legitimately worthy of consideration by physicists.

At the very least we hope to convince you that a lot of interesting ideas arise in the context of these questions.
These include the following:
Hamiltonian systems have strong symmetries that make some kinds of simulations natural, and others impossible.
There is finite computing capacity inside our Universe, and thus there are limits to what kinds of simulations would be possible \emph{by} us, and this capacity changes very dramatically if we ourselves are inside an approximate simulation.
Good (and bad) software engineering practices might have implications for observables.
The epistemological basis for cosmological measurement and discovery---which is already problematic---gets even weirder if we consider the possibility that we live inside an engineered system.
We work through these ideas and more, discursively, in what follows.
We attempt to gather together our main predictions or conclusions in \secref{sec:discussion}.

\section{Hamiltonian systems}

It is possible to think of the Universe as being a very big, analog computer.
Every thing that happens is computed, analytically, by the laws of physics.
This way of thinking is strongly encouraged, in some sense, by the symmetries of Hamiltonian systems:
In a Hamiltonian system, the physicist has a huge range of choices to make about the coordinate system, such that different observers might give very different---but all equally valid---descriptions of the same system.

... different canonical representations are simulating each other, exactly! Cite things.

... True for quantum mechanics too, which is also hamiltonian. Cite Feynman things.

So we DO live in a simulation, in the sense that the laws of physics that operate here are probably subject to exact transformations to other extremely different representations.

By the way, maybe this relates to the low entropy of the initial conditions of the Universe...?

\section{What is a simulaton?}

The above considerations about Hamiltonian systems lead to the question: What do we mean when we propose that we might be living in a simulation?

If we live in an isolated hamiltonian system that is not being observed or interacted with from the outside (as it were), there will be no observational consequences to the hypothesis that it is a simulation.
Indeed, for many physicists, this is precisely how we \emph{do} think of our Universe.

I think it is only an interesting hypothesis when it means that there is an engineered or designed physical computing system inside of which we live, and that the externals are at least occasionally observing it or interacting with it.

And it is even more interesting if it involves physical approximations or if the computation has strong limitations.

\section{Computing capacity of the Universe}

\section{Quantum computing}

\section{Classical approximate simulations}

\section{The problem of patching}

\section{Simulations within simulations}

\section{The intentions and practices of the externals}

Why are we being simulated? Because those parts will be best, we expect.

Bugs and Mandela effect.

Towards the end of this section, conclude that, within the simulation hypothesis, the epistemological status of god changes.

\section{Discussion and conclusions}\label{sec:discussion}

\paragraph{Acknowledgements:}

\section*{References}

\end{document}

%%
OLD STUFF FOLLOWS
%%

\section{The plausibility}

ALL WRONG / FIX....students, especially undergraduates interested in majoring in physics.
It is also a question that is (effectively) asked by many popular
movies, including \textit{The Matrix} (DATE), \textit{Total Recall}
(DATE) and others.
It seems ridiculous to even contemplate it, but then---as one does
contemplate it---it becomes clear that it is hard to \emph{rule out}
empirically or observationally.
What would it be like to live in a simulation, and how would it be
different from---or the same as---what we currently observe?
The question has religious connotations or connections too, because if
it turns out to be true that we do live in a simulation, then it might
turn out to be true, strictly speaking, that we have a \emph{maker}.
As usual, however, when physics questions meet religious questions, it
will turn out
physics doesn't provide the relevant answers.

One of the remarkable things about physical law, at it's most fundamental,
is that if the world is unitary and Hamiltonian, then there is a very
true sense that we could be an a perfect analog-computer simulation.
In both classical mechanics (\eg, CITE GOLDSTEIN) and quantum mechanics
(\eg, CITE SOMETHING), there are perfect symplectic coordinate transformations
between Hamiltonian systems, such that any Hamiltonian system (possibly subject
to some weak assumptions like a time-independent form for the Hamiltonian)
can be transformed into any other Hamiltonian system, delivering identical
dynamics.

When these transformations are made, combinations of positions and
velocities from one coordinate system or representation become
positions in the other coordinate system, and different combinations
become velocities....
Initial conditions...
Entropy and etc...

... the point that this means that the question has to become a
statistical or probabilistic one.

\section{The implausibility}

(Introduce spacetime volume and mass-time ideas. Call out the point
that within general relativity there is a measure problem that might
make what's being said here technically wrong, but you know what I
mean (sort of?). Woah, maybe this could be used to define a measure??)

\section{The high probability}

Famously, Bostrom (2003) solves this implausibility argument by saying
that \emph{obviously} extremely intelligent and technologically
capable entities will make ``ancestor simulations'' that have the
property that they are very approximate, but self-correcting: If an
intelligent agent within the simulation discovers that they are in the
simulation, the simulation will be run back and re-started with that
part simulated at higher resolution (so the intelligent agent never
makes any such discovery). Very clever! It permits a technological entity
to make a simulation that is undetectable by any agent within it, without
requiring simulation hardware that is larger in spacetime volume than
the world it simulates.

Bostrom goes on to say that if it is natural for any very
technologically capable entities to make such simulations, and if
those simulations are rich enough that the agents within them will
become technologically capable enough to make such simulations, then
they will also make rich simulations, and so on. Once you build this
tower of simulations, he claims, we are exceedingly unlikely to be at
the top level of the tower.
That is, most intelligent creatures in the full stack
of universes (top-level plus all simulations) will end up \emph{not}
being in that top-level universe. Good argument. Or is it?

In addition to the usual eye-rolling that these arguments receive, I
would highlight two extremely weak assumptions:

The first weakness is that, ultimately, all of these simulated
universes are being simulated by the top-level hardware. That is,
every move that is made by every agent in every simulation within
every simulation within every simulation of this simulation stack is
being computed by the hardware at the top level. Even if this
top-level hardware is very sophisicated---even if the entities
operating it have commandeered the vast majority of the computing
available to them on their planetary system which itself is made
entirely of computers, there will have to be incredibly effective
speed-ups and approximations going on in the simulations to make this
all possible.

One way to think about it is this: Imagine that the simulations are
happening with an efficency factor of $q$: That is, because of amazing
speed-ups obtained by performing the simulations sufficiently
approximately, every tonne-hour of computing at the top level can
compute $q>1$ tonne-hours of world-simulation in the stack of
simulated worlds, on average. Then technologically-capable
civilizations that want to run such simulations will put far more
intelligent agents into the simulations and sub-simulations than exist
at the top-level world when (very roughly speaking) they apportion far
more than $1/q$ of their total world (like their homes, biosphere,
computers, land, vehicles, and so on) into performing such
simulations. That seems like a lot! Even if you got $q$ close to 100,
which seems implausible, would you really use 1 percent of your total
resources (and I don't just mean compute resources, I mean
\emph{total} resources including all equipment, energy, food,
personnel, and so on) on this project? A project that only addresses a
small fraction of all interesting non-monetizable science questions?
That's a pretty strong assumption.

Bostrom considers it obvious that enormously expensive simulation
efforts would happen at every level in the hierarchy of
simulations. But I don't think it would happen at \emph{any} level in
the hierarchy. Unless the singularity (\eg, Good 1966, or Chalmers 2010) leaves us with
machines-in-charge that are way more interested in pure science than
we are! That is, the fact that there is a hierarchy of simulations
does not escape the problem that each simulation of every action in
every simulated universe puts compute requirements (in some product of
computing mass times time or total computing energy, say) at the top
level. Those requirements are finite, and probably killing.

The second weakness I'd like to point out is the unexamined premise
that you could just stop the simulation and reverse it when someone
inside the simulation discovers that they are in a simulation! CITE
WITTGENSTEIN AND THOMPSON.

\section{My attitude, and implications}

Comments on Hinchliffe's Law.

Does Bostrom's paper already rule out the hypothesis? No, because it
is a bad paper.

Comments on the existence of God.

\section*{References}
\begin{trivlist}
\item Bostrom,~N., 2003, Are You Living in a Computer Simulation?,
  \textit{Philosophical Quarterly} \textbf{53} 211 243--255.
\item Chalmers,~D.~J., The Singularity: A philosophical analysis,
  \textit{Journal of Consciousness Studies} \textbf{17} 7--65.
\item Good,~I.~J., 1966. Speculations concerning the first ultraintelligent machine,
  in (F.~L.~Alt \& M.~Rubinoff, eds.) \textit{Advances in Computers} \textbf{6} 31--88.
\item \textit{Matrix, The}, HOGG DATE ETC.
\item Thompson, A., 2017, Scientists Can Now Read Your Thoughts With a Brain Scan,
  \textit{Popular Mechanics} 2017 June 27.
\item \textit{Total Recall}, HOGG DATE ETC.
\item Wikipedia editors, Betteridge's law of headlines,
  \url{https://en.wikipedia.org/wiki/Betteridge%27s_law_of_headlines}
    (accessed 2020 March 28)
\item Wittgenstein, L., 1958, \textit{The Blue and Brown Books},
  Basil Blackwell.
\end{trivlist}

\end{document}
