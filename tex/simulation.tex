\documentclass[12pt,letterpaper]{article}
\usepackage{url}

% adjust typesetting
\addtolength{\topmargin}{-0.75in}
\addtolength{\textheight}{1.4in}
\linespread{1.08333} % 10/13 spacing

\begin{document}

\section*{Do we live inside a computer simulation?}

\noindent
\textbf{David W. Hogg}\\
\footnotesize{Center for Cosmology and Particle Physics, Department of Physics, New York University}

\paragraph{Abstract:}
I am frequently asked this question, and I have three-ish things to
say about it:
The first is that Hamiltonian systems have remarkable transformation
properties, such that any Hamiltonian system can be used to
more-or-less exactly simulate any other Hamiltonian system.
In this sense, we \emph{are} in a simulation: Any Hamiltonian system
is a simulation of any other Hamiltonian systems at the same number of
degrees of freedom.
This is true in either quantum mechanics or classical mechanics.
The second thing is that if the simulation in which we live is
\emph{exact}, and if it is deliberately being run by some intelligence
in some meta-universe, and if that simulation has input and output
devices and so on, then it is overwhelmingly unlikely that we
currently live in a simulation.
The third thing is that there is an argument from philosophy that we
are very likely to be living in a simulation (or, actually, the claim
is of a trilemma).
This philosophical argument relies on some not-tenable assumptions,
including one about being able to know what creatures know, which is
by no means obviously possible (and quite impossible, right now, by
us, in this Universe).
One interesting connected issue is that if we do indeed live in a
computer simulation of the kind envisioned by movies and philosophers,
then quite literally there is a god or gods.
It is up to the reader to decide if this paper is compliant with
Hinchliffe's Law.

\section{The plausibility}

\section{The implausibility}

\section{The high probability}

\section{My attitude, and implications}

\section*{References}
\begin{trivlist}
\item Bostrom, N., 2003, ``Are You Living in a Computer Simulation?'',
  \textit{Philosophical Quarterly} \textbf{53} 211, 243--255.
\item Wikipedia editors, ``Betteridge's law of headlines'',
  \url{https://en.wikipedia.org/wiki/Betteridge%27s_law_of_headlines}
    (accessed 2020 March 28)
\item Wittgenstein, L., 1958, \textit{The Blue and Brown Books},
  Basil Blackwell.
\end{trivlist}

\end{document}
