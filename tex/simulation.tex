\documentclass[12pt,letterpaper]{article}
\usepackage{url}

% adjust typesetting
\addtolength{\topmargin}{-0.95in}
\addtolength{\textheight}{1.8in}
\linespread{1.08333} % 10/13 spacing

% text macros
\newcommand{\documentname}{\textsl{Note}}

\begin{document}

\section*{Do we live inside a computer simulation?}

\noindent
\textbf{David W. Hogg}\\
{\footnotesize{Center for Cosmology and Particle Physics, Department of Physics, New York University}}

\paragraph{Abstract:}
I am frequently asked this question, and I have three-ish things to
say about it:
The first is that Hamiltonian systems have remarkable transformation
properties, such that any Hamiltonian system can be used to
more-or-less exactly simulate any other Hamiltonian system.
In this sense, we \emph{are} in a simulation:
Any Hamiltonian system can be seen as an analog computer that is
simulating many other Hamiltonian systems at the same number of
degrees of freedom.
This is true in either quantum mechanics or classical mechanics.
The second thing is that if the simulation in which we live is
\emph{exact}, and if it is deliberately being run by some intelligence
in some meta-universe, and if that simulation has input and output
devices and so on, then it is overwhelmingly unlikely that we
currently live in a simulation.
The third thing is that there is an argument from philosophy (Bostrom)
that we are very likely to be living in a (very approximate but
cleverly patched) simulation.
This philosophical argument relies on some very weak assumptions,
including one about being able to know what creatures know, which is
by no means obviously possible (and quite impossible, right now, by
us, in this Universe).
One interesting connected issue is that if we do indeed live in a
computer simulation of the kind envisioned by movies and philosophers,
then quite literally there is a god or gods.
This \documentname\ is compliant with Hinchliffe's Law.

\section{The plausibility}

Do we live in a simulation? This question is asked of me frequently by
students, especially undergraduates interested in majoring in physics.
It is also a question that is (effectively) asked by many popular
movies, including \textit{The Matrix} (DATE), \textit{Total Recall}
(DATE) and others.
It seems ridiculous to even contemplate it, but then---as one does
contemplate it---it becomes clear that it is hard to \emph{rule out}
empirically or observationally.
What would it be like to live in a simulation, and how would it be
different from---or the same as---what we currently observe?
The question has religious connotations or connections too, because if
it turns out to be true that we do live in a simulation, then it might
turn out to be true, strictly speaking, that we have a \emph{maker}.
As usual, however, when physics questions meet religious questions, it
will turn out (despite wrong claims by Dawkins, DATE, and others) that
physics doesn't provide the relevant answers.

\section{The implausibility}

(Introduce spacetime volume and mass-time ideas.)

\section{The high probability}

Famously, Bostrom (2003) solves this implausibility argument by saying
that \emph{obviously} extremely intelligent and technologically
capable entities will make ``ancestor simulations'' that have the
property that they are very approximate, but self-correcting: If an
intelligent agent within the simulation discovers that they are in the
simulation, the simulation will be run back and re-started with that
part simulated at higher resolution (so the intelligent agent never
makes any such discovery). Very clever! It permits a technological entity
to make a simulation that is undetectable by any agent within it, without
requiring simulation hardware that is larger in spacetime volume than
the world it simulates.

Bostrom goes on to say that if it is natural for any very
technologically capable entities to make such simulations, and if
those simulations are rich enough that the agents within them will
become technologically capable enough to make such simulations, then
we are exceedingly unlikely to be at the top level of this hierarchy
of simulations. That is, most intelligent creatures in the full stack
of universes (top-level plus all simulations) will end up \emph{not}
being in that top-level universe. Good argument. Or is it?

In addition to the usual eye-rolling that these arguments receive, I
would highlight two extremely weak assumptions:

The first weakness is that, ultimately, all of these simulated
universes are being simulated by the top-level hardware. That is,
every move that is made by every agent in every simulation within
every simulation within every simulation of this simulation stack is
being computed by the hardware at the top level. Even if this
top-level hardware is very sophisicated---even if the entities
operating it have commandeered the vast majority of the computing
available to them on their planetary system which itself is made
entirely of computers, there will have to be incredibly effective
speed-ups and approximations going on in the simulations to make this
all possible.

One way to think about it is this: Imagine that the simulations are
happening with an efficency factor of $q$: That is, because of amazing
speed-ups obtained by performing the simulations sufficiently
approximately, every tonne-hour of computing at the top level can
compute $q>1$ tonne-hours of world-simulation in the stack of
simulated worlds, on average. Then technologically-capable
civilizations that want to run such simulations will put far more
intelligent agents into the simulations and sub-simulations than exist
at the top-level world when (very roughly speaking) they apportion far
more than $1/q$ of their total world (like their homes, biosphere,
computers, land, vehicles, and so on) into performing such
simulations. That seems like a lot! Even if you got $q$ close to 100,
which seems implausible, would you really use 1 percent of your total
resources (and I don't just mean compute resources, I mean
\emph{total} resources including all equipment, energy, food,
personnel, and so on) on this project? A project that only addresses a
small fraction of all interesting non-monetizable science questions?
That's a pretty strong assumption.

Bostrom considers it obvious that enormously expensive simulation
efforts would happen at every level in the hierarchy of
simulations. But I don't think it would happen at \emph{any} level in
the hierarchy. Unless the singularity leaves us with
machines-in-charge that are way more interested in pure science than
we are! That is, the fact that there is a hierarchy of simulations
does not escape the problem that each simulation of every action in
every simulated universe puts compute requirements (in some product of
computing mass times time or total computing energy, say) at the top
level. Those requirements are finite, and probably killing.

The second weakness I'd like to point out is the unexamined premise
that you could just stop the simulation and reverse it when someone
inside the simulation discovers that they are in a simulation! CITE
WITTGENSTEIN AND THOMPSON.

\section{My attitude, and implications}

Comments on Hinchliffe's Law.

Does Bostrom's paper already rule out the hypothesis? No, because it
is a bad paper.

Comments on the existence of God.

\section*{References}
\begin{trivlist}
\item Bostrom, N., 2003, ``Are You Living in a Computer Simulation?'',
  \textit{Philosophical Quarterly} \textbf{53} 211 243--255.
\item Dawkins, R., HOGG DATE ETC.
\item \textit{Matrix, The}, HOGG DATE ETC.
\item Thompson, A., 2017, ``Scientists Can Now Read Your Thoughts With a Brain Scan'',
  \textit{Popular Mechanics} 2017 June 27.
\item \textit{Total Recall}, HOGG DATE ETC.
\item Wikipedia editors, ``Betteridge's law of headlines'',
  \url{https://en.wikipedia.org/wiki/Betteridge%27s_law_of_headlines}
    (accessed 2020 March 28)
\item Wittgenstein, L., 1958, \textit{The Blue and Brown Books},
  Basil Blackwell.
\end{trivlist}

\end{document}
