% Copyright 2021 the authors. All rights reserved.

% To-do items
% -----------
% - make sure I make the point that the sim hypothesis DOES make predictions if there is any finite probability on the simulation being imperfect
% - make sure I make the point that this, in turn, makes the existence of God potentially observable
% - make sure I clearly make the distinction between a perfect simulation and an approximate simulation
% - make sure I clearly make the distinction between an intentionally constructed simulation and a naturally occurring simulation, if that distinction can exist?
% - connection to the anthropic principle: The Universe has a purpose. So maybe the anthropic principle is different in a simulated universe?
% - the point that in principle, one generic prediction of being in a simulation is that the world would then be simulate-able itself. This prediction is confirmed in our universe. Cite TikToker @briicstar01.

\documentclass[letterpaper]{article}
\usepackage{url}
\usepackage{xcolor}

% adjust typesetting and page layout
\makeatletter
\renewcommand\section{\@startsection {section}{1}{\z@}%
  {-3.25ex \@plus -1ex \@minus -.2ex}%
  {1.5ex \@plus .2ex}%
  {\raggedright\normalfont\large\bfseries}}
\makeatother
\addtolength{\topmargin}{-0.90in}
\setlength{\textheight}{2\textwidth}
\linespread{1.08333} % 10/13 spacing
\pagestyle{myheadings}
\markright{\textsf{\upshape\color{gray}%
Hogg \& Seraphim --- Is the simulation hypothesis a physics question?}\hfill}
\thispagestyle{empty}
\raggedbottom\sloppy\sloppypar\frenchspacing

% text macros
\newcommand{\documentname}{\textsl{Note}}
\newcommand{\sectionname}{Section}
\newcommand{\secref}[1]{\sectionname~\ref{#1}}
\newcommand{\foreign}[1]{\textsl{#1}}
\newcommand{\eg}{\foreign{e.~g.}}

\begin{document}

\section*{Does the simulation hypothesis lead to physics questions?}

\medskip\noindent
\textbf{David W. Hogg}
{\par\noindent\footnotesize%
\url{https://orcid.org/0000-0003-2866-9403}\\Center for Cosmology and Particle Physics, Department of Physics, New York University%
\par}

\medskip\noindent
\textbf{Paula Seraphim}
{\par\noindent\footnotesize%
College of Arts and Sciences, New York University%
\par}

\paragraph{Abstract:}
Do we live inside a computer simulation?
This question is of interest to many students of physics, but is it really a physics question?
One possible answer is that if the simulation hypothesis makes predictions---even vague or conditional predictions---for the outcomes of experimental physics programs, then yes, it is a physics question.
There are indeed a number of such predictions, depending on whether the simulation is classical or quantum, how the simulators or simulating universe intervenes in the simulation, what kinds of approximations are in play, and other engineering considerations (such as bugs).
For example, if the simulation is, or involves, exact quantum simulation of quantum states in this world, and it is being observed by (or interacting with) any external system, then there might be observable violations of unitarity.
For another, if the simulation is approximate or spatially limited, the limits of the approximations might lead to observable effects.
No particular observable consequences of the simulation are predicted at high confidence because they are all conditional on details of the simulation technology and objectives, but there are nonetheless generic signatures that could be used to guide experimental searches; in this way the search for evidence for a simulation is somewhat akin to searches for the particle nature of the dark matter.
One epistemological point is that if we do indeed live in a computer simulation of the kind envisioned by movies and philosophers, then quite literally our maker (something like god) is potentially subject to discovery by experimentation.
This \documentname\ violates Hinchliffe's Law.

\section{Introduction}

When one spends some time at the beach, watching the waves slowly break and crash onto the sand in foamy washes, with sandpipers running to grab tiny insects displaced by the water in the sand, one must to conclude that the external simulators---if we do indeed live in a computer simulation---are spending one ridiculous amount of computing on the shoreline.
Or maybe the externals are only spending their computation on the part of the shoreline that one is currently looking at, and only for that time, but still?
Or maybe it is a full quantum simulation, in which case the externals are spending way more computation on \emph{each individual grain of sand} than whatever is required to make those waves crash and sandpipers run.
But of course we don't know the intentions and capabilities of the externals.
Maybe computation is really cheap in the external world, and maybe the questions being answered by this simulation (the one we are in) are of enormous importance out there.
The question arises: Is it conceivable that we live in a computer simulation?
And what do we even mean by that idea?

The point of this \documentname{} is \emph{not} to answer the question ``Do we live in a simulation?''.
It is to answer the question ``Is this a physics question, worthy of consideration by physicists?''.
Oddly, we think the answer to this latter question is ``yes'':
The simulation hypothesis is important enough to justify consideration, and it can and does lead to some natural kinds of (perhaps vague) predictions for physical experiments in physics laboratories and observatories.
That is, the simulation hypothesis is an idea that can inspire searches for new physics or anomalies, in much the same way that other high-level ideas in physics, such as inflation, string theory, the vacuum energy, or unification have inspired and continue to inspire experimental searches.

Here we discuss some of the relevant questions and definitions, with a goal of starting this conversation.
Because we encounter these questions so much in the undergraduate-education context, we have tried to set the level for an advanced undergraduate studying physics, philosophy, neuroscience, or a related field.
We are aware that many physicists consider the simulation hypothesis ridiculous or fantastical.
But our view is that things that are legitimately interesting to students of physics are themselves legitimately worthy of consideration by physicists.

At the very least we hope to convince you that a lot of interesting ideas arise in the context of these questions.
These include the following:
Hamiltonian systems have strong symmetries that make some kinds of simulations natural, and others impossible.
There is finite computing capacity inside our Universe, and thus there are limits to what kinds of simulations would be possible \emph{by} us, and this capacity changes very dramatically if we ourselves are inside an approximate simulation.
Good (and bad) software engineering practices might have implications for observables.
The epistemological basis for cosmological measurement and discovery---which is already problematic---gets even weirder if we consider the possibility that we live inside an engineered system.
We work through these ideas and more, discursively, in what follows.

One of our main contributions is to classify simulation scenarios or hypotheses, in a physical sense; we do that in \secref{sec:classification}.
We attempt to gather together our main predictions or conclusions in \secref{sec:discussion} and we make some comments on the literature in the annotated bibliography in \secref{sec:bibliography}.

\section{Hamiltonian systems}\label{sec:hamiltonian}

It is possible to think of the Universe as being a very big, analog computer.
Every thing that happens is computed, analytically, by the laws of physics.
This way of thinking is strongly encouraged, in some sense, by the symmetries of Hamiltonian systems:
In a Hamiltonian system, the physicist has a huge range of choices to make about the coordinate system or representation, such that different observers might give very different---but all equally valid---descriptions of the same system.

For an extremely simple example: In the two-body problem in which two charges orbit one another in electromagnetism (or two masses in gravity), the system can be analyzed in terms of the positions of the two particles and their velocities, or in terms of the position of the center of mass and the displacement between the particles and the time derivatives of those vectors.
This example is very simple!
But there is no truth to the matter of which representation is more fundamental or more correct:
Both representations lead to sensible, unitary, Hamiltonian dynamics.
Thus there is no sense in which the representation that we think of as being fundamental is the only valid description of the physics.
This physics might emerge from another representation that is extremely different, and possibly far more complex (or far simpler).

Because there are these representation transformations that preserve all physical character, but look very different, there is a very real sense in which we might live in some kind of \emph{exact} simulation.
In this context, there is one kind of somehow fundamental physics, and the physics we experience and observe is some canonical tranformation thereof.
These kinds of exact simulations are possible in both quantum and classical Hamiltonian systems; it is controlled by the point that they are Hamiltonian, not whether they are classical or quantum.
Indeed, these observations have been made in literature from the 1980s \cite{feynman, feynman2} that was preface to the literature on quantum computing, where the original ideas were to make laboratory quantum systems that could exactly simulate other or natural quantum systems.

So we DO live in a simulation, in the sense that the laws of physics that operate here are probably\footnote{Some would say ``definitely'' and not ``probably'' here, but it is possible that Hamiltonian symmetry is an approximate symmetry in our Universe.} subject to exact transformations to other extremely different representations.

...By the way, maybe this relates to the low entropy of the initial conditions of the Universe...?

Of course if this Hamiltonian system is being observed by externals, it will show departures from unitarity. Why? Because it must then excite degrees of freedom outside our Universe. So in this case it must be that the computer isn't truly Hamiltonian, but a non-Hamiltonian system approximating a Hamiltonian system, or else we should see departures from unitarity in sufficiently sensitive and appropriate physics experiments.

\section{What is a simulation?}\label{sec:classification}

The above considerations about Hamiltonian systems lead to the question: What do we mean when we propose that we might be living in a simulation?

If we live in a naturally occurring (that is, not engineered) isolated Hamiltonian system that is not being observed or interacted with from the outside (as it were), there will be no observational consequences to the hypothesis that it is a simulation, for the reasons discussed in \secref{sec:hamiltonian}.
Indeed, for many physicists, this is precisely how we \emph{do} think of our Universe:
a naturally occurring Hamiltonian system with no external interactions!
For our purposes, we won't consider this case as a simulation at all.

Another, related case that we won't consider to be a simulation is if this Universe is a naturally occurring hamiltonian system that is interacting with an external universe but that interaction is confined to extremely distant space-time points, which are outside our horizon (or near our horizon).
This is the case, for instance, if---as seems likely---our Universe is a bubble within a larger, inflating meta-universe filled with many bubbles.
In this scenario, our Universe does interact (weakly) with its surroundings (see, for example, \cite{kleban}), but it is not, for our purposes, a simulation, and there are no local observational consequences of the interactions, or only \emph{exceedingly indirect consequences}.

The simulation hypothesis is only an interesting hypothesis when it means that there is an engineered or designed or observed or \emph{at least} locally externally interacting physical computing system inside of which we live.
One minimal simulation hypothesis is that our Universe is interacting locally with another external physical system that is fully (except for the interaction channel) outside this Universe.
By ``local'' here we mean that the interaction with the external system or Universe is not only at some extremely distant boundary (and especially not outside our horizon).

And it is even more interesting if it involves physical approximations or if the computation has strong limitations.

DEFINE TERMS, like give names for the different cases, and give names to the external universe and the externals and the intentional simulators and so on. Also the concept of four-volume of the simulation. Is this covariant?

BREAKDOWNS INCLUDE: approximate or exact. observed or not observed. solipsistic or multi-player. involving humans (or human parts) or all-computation. adversarial or not.

\section{Computing capacity of the Universe}

If our Universe is being exactly simulated [use consistent terminology here], then the Universe in which it is being simulated must have, in some sense, a higher overall capacity for computation than this Universe.
That is, if there is anything \emph{else} going on in the external Universe, this simulation must not be using up \emph{all} of its computing capacity.
Of course our simulation could be running at low speed (time could be passing much more slowly for us than for the externals), to save compute capacity.
But even then, if we think of the four-volume (spatial volume times time duration) taken up by our simulation in the external world, it must be larger than the four-volume of computation available in our Universe.

Even if our Universe is a very approximate simulation, there are limits on the kind of computer that might be hosting this simulation, and the runtime on that computer. .  .


\section{Quantum computing}

\section{Classical approximate simulations}

The Thirteenth Floor, The Matrix, Total Recall.

\section{The problem of patching}

The importance of patching to Bostrom's paper. If done perfectly, it ensures that the simulation has no observable consequences. It is what truly deceptive, approximate simulators would do.

It is only one option, however, so it doesn't make the simulation hypothesis unobservable; it just means that there are simulation methods that would effectively render the simulation unobservable. It's hard or impossible to put probabilities on that.

The fact that patching isn't obviously possible. It relies on some things in neuroscience I think maybe?

\section{Brains in vats}

Like Matrix.

Also the case where you, reading this article, are the only intelligence in the whole Universe.

\section{Simulations within simulations}

The Bostrom argument relies on this. Maybe Kipping too?

It is very unlikely that you can do an approximate simulation within an approximate simulation, because you wouldn't be able to do it cheaply, at the upper level. This is a bit of a subtle argument.

\section{The intentions and practices of the externals}

Why are we being simulated? Because those parts will be best, we expect.

Bugs and Mandela effect.

Joke about the idiocy of the Matrix justification.

Towards the end of this section, conclude that, within the simulation hypothesis, the epistemological status of god changes.

\section{Discussion and conclusions}\label{sec:discussion}

\paragraph{Acknowledgements:} DWH would like to thank the generations of students at New York University who have asked and debated these questions in classes and office hours.
It is a pleasure to thank Matt Kleban (NYU) and Melissa Ness (Columbia) for valuable conversations and comments. This work was partially supported by HOGG GRANT NUMBERS HERE.

\section{Annotated bibliography}\label{sec:bibliography}
\nocite{*}
\renewcommand{\section}[2]{} % extreme HACK
\bibliographystyle{hacked-annote}
\bibliography{simulation}

\end{document}